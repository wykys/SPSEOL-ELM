\section{Vzory výpočtů}
  
  Výpočet relativní procentuální chyby digitu:
  \begin{equation}
    \delta _{digit\%} = \dfrac{\pm digit}{MH} \cdot 100 = \dfrac{\pm 0,01}{0,81} \cdot 100 \doteq \underline{\underline{\pm 1,234~\%}}
    \nonumber
  \end{equation}
  
  Celková procentuální chyba:
  \begin{equation}
    \delta_{\%} = \pm\delta_{MH\%} \pm \delta_{digit\%} = \pm 0,5 \pm 1,234 \doteq \underline{\underline{\pm 1,734~\%}}
    \nonumber
  \end{equation}
  
  Celková absolutní chyba:
  \begin{equation}
    \Delta U = \dfrac{\delta_\%}{100} \cdot MH = \dfrac{0,859}{100} \cdot 16,973 = \underline{\underline{\pm 145,784~\Omega}}
    \nonumber
  \end{equation}
  
  Výpočet teoretické hodnoty výstupního napětí s využitím vztahu (4)
  \begin{equation}
    U_{OUT_{VYP}} = A_N \cdot U_{IN} = -\dfrac{R_2}{R_1} \cdot U_{IN} =  -\dfrac{5}{5} \cdot 0,8 = \underline{\underline{-0,8~V}}
    \nonumber
  \end{equation}
  
  Výpočet $\Delta U_{OUT} - U_{OUT_{VYP}}$
  \begin{equation}
    \Delta U_{OUT} - U_{OUT_{VYP}} = U_{OUT} - U_{OUT_{VYP}} = -0,81 - (-0,8) = \underline{\underline{0,01~V}}
    \nonumber
  \end{equation}
  
  Výpočet rezistorů $R_1$ a $R_2$ pro zisk $20~dB$
  \begin{eqnarray}
    20 &=& 20 \cdot \log\dfrac{U_{OUT}}{U_{IN}} \\\nonumber
    1 &=& \log\dfrac{U_{OUT}}{U_{IN}} \\\nonumber
    \log10 &=& \log\dfrac{U_{OUT}}{U_{IN}} \\\nonumber
    10 &=& \dfrac{U_{OUT}}{U_{IN}} = -\dfrac{R_2}{R_1} \\\nonumber
    R_1 &=& \underline{\underline{10~k\Omega}} \\\nonumber
    R_2 &=& 10 \cdot R_1 = 10 \cdot 10 = \underline{\underline{100~k\Omega}}
    \nonumber
  \end{eqnarray}
