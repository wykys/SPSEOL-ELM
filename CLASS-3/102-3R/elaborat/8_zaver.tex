\section{Závěr}
  
%  \begin{tabular}[H]{rcrl}
%    $20~V$ & $\pm0,5\%$ z MH $\pm 1$ digit & $19.99~V$ & $0,01~V$ \\
%    $2~mA$ & $\pm0,8\%$ z MH $\pm 1$ digit & $1.999~V$ & $0,001~mA$ \\
%    $20~mA$ & $\pm0,8\%$ z MH $\pm 1$ digit & $19.99~V$ & $0,01~mA$ \\
%    $200~mA$ & $\pm1,5\%$ z MH $\pm 5$ digit & $199.9~V$ & $0,1~mA$
%  \end{tabular}
  
  \subsection{Chyby měřících přístrojů}
    \indent\indent
    Procnruální chyba použitých měřících přístrojů ($M_1$ a $M_2$) nepřekrožila Při měření stejnosměrných napětí $2~\%$ a pri měření srřídavých napětí $1~\%$, tidíž by se dali považovat použité měřící přístroje za vhodné a naměřené hodnoty za dostatečně přesné. Maximální procentuální chyba byla pri měření stejnosměrných napětí $1,73~\%$, pri měření střídavých napětí myximální procentuální hyba dosáhla hodnoty $0,85~\%$.
  
  \subsection{Zhodnocení}
    \begin{enumerate}
      \item
        Zjistil jsem záhladní parametry OZ MAA741 a jeho klíčové parametr jsem shrnul do tabulky č. 1, dále jsem namaloval schéma vnitřního zapojení tohoto obvodu, které je rovněž obsaženo v úvodu a to pod označením schéma č. 3.
      \item
        Změřil jsem závislos výstupního napětí na rezistoru $R_2$. Tuto závislost jsem zakreslil do grafu.
      \item
      	K naměřeným hodnotám jsem vypočítal teoretické hodnoty za pomoci vztahů které jsem odvodil v úvodu. Dále jsem spočítal Abdolutní odchylku naměřené honoty výstupního napětí od hosnoty teoretické, největší odchylka dosáhla $2,7~V$.
      \item
        Změřil jsem zývislost výstupního napětí na rezistoru $R_1$ při stejnosměrném buzení. Naměřené hodnoty byli opět vyneseni do grafu.
      \item
        Vypočítal jsem hodnoty rezistorů tak, aby výstupní napětí bylo desetkráv vyšší než napětí vstupní, tedy zisk o $20~dB$. Změřil jsem OZ v invertujícím zapojení s těmito rezistory a výsledné průběhy zakreslil do grafu č. 3.
    \end{enumerate}
