\begin{minipage}[H][11.48cm][c]{0.8\textwidth}
  \begin{enumerate}
    \item
      Seznsamte se s důležitými katalogovými údaji měřeného OZ, údaje vypište do přehledné tabulky.
    \item
      Změřte a nakreslete závislosti výstupního napětí invertujícího zesilovače s OZ na zpětnovazebném rezistoru $U_{OUT} = f(R_2)$ při stejnosměrném vstupním napětí $U_{IN} = 0,8~V$ a vstupním odporu $R_1 = 5~k\Omega$.
    \item
    	Pro pědchozí měření vypočítejte teoretické hodnoty výstupního napětí $U_{OUT_{VYP}}$. Naměřené a vypočítané hodnoty srovnejte a vypočítejte jejich odchylku.
    \item
      Měřením ověřte činnost OZ pracujícího jako invertující zesilovač harmonického vstupního napětí $U_{IN_{RMS}} = 0~V$, $U_{IN_{AVG}} 0,8~V$, $f = 5~kHz$, při zětnovazebném rezistoru $R_2 = 10~k\Omega$.
    \item
      Pro vstupní harmonické napětí (z bodu 4) a napěťový přenos zesolovače $a_u = 20~dB$ změřte $U_{OUT_{RMS}}$, $U_{OUT_{AVG}}$, $f$ a zakreslete časové průběhy vstupního a výstupního napětí zesilovače.
  \end{enumerate}

\end{minipage}


