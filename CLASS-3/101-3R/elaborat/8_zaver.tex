\section{Závěr}
  
%  \begin{tabular}[H]{rcrl}
%    $20~V$ & $\pm0,5\%$ z MH $\pm 1$ digit & $19.99~V$ & $0,01~V$ \\
%    $2~mA$ & $\pm0,8\%$ z MH $\pm 1$ digit & $1.999~V$ & $0,001~mA$ \\
%    $20~mA$ & $\pm0,8\%$ z MH $\pm 1$ digit & $19.99~V$ & $0,01~mA$ \\
%    $200~mA$ & $\pm1,5\%$ z MH $\pm 5$ digit & $199.9~V$ & $0,1~mA$
%  \end{tabular}
  
  \subsection{Chyby měřících přístrojů}
    \indent\indent
    Procnruální chyba použitých měřících přístrojů nepřekrožila $1~\%$ a tudíž by jse změřené hodnoty dali považovat za relativně správné.
  
  \subsection{Zhodnocení}
    \begin{enumerate}
      \item
        Úspěšně jsem navrhnul generátor obdélníkových napětí s IO555, na jeho výstupu sice nejsou ideální obdélníky, ale to není mou chybou, to je důkazem že žijeme v nedokonalém světě, kde se věci pohybují skokově jen na mikroskopické úrovni.
      \item
        Změřil jsem hodnoty použitých součástek a spočítal jejich procentuální a absolutní chyby. nejpčesněj měřil přístroj RLCG BM 959 s procentuální chybou rovnou 0,175~\%.
      \item
        Pomocí přípravku jsem sestavil generátor a změřil veličiny dle zadání. Výsledky jsou shrnuty v tabulce č. 2. Dobu sestupné hrany nebylo možné změřit, protože osciloskop ukazoval zápozný čas, což je absurdní. Námežná hrana byla zmeřena na $4~\mu s$.
      \item
        Změřil jsem dobu nabíjení a vybíjení kondenzátoru a došel jsem k závětu že doba nabíjení je téměř totožná s úrovní H na pinu 3 a doba vybíjení je témeř identická z dobou úrovně L na pinu 3. Minimální hodnota napětí byla změřena na $2~V$ což odpovídá $\frac{1}{3}U_{CC}$ a maximální hodnota napětí byla se rovná $4~V$ což odpovídá $\frac{2}{3}U_{CC}$.
      \item
        Na PC jsem vytvořil grafy napěťových průběhé na pinu 2 a 3 IO555.
    \end{enumerate}
