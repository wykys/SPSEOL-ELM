\section{Postup měření}
  \subsection{Měření rezistorů R$_1$ a R$_2$}
    \begin{itemize}
      \item
        Zapneme DMM a nastavíme vhodný rozsah.
      \item
        Vezmeme rizistor R$_1$ a vložíme jej do svorek DMM, dáváme pozor, abychom se nedotýkali přívodních vodičů rezistoru, čímžbychom zkreslovali výsledky měření.
      \item
        Poznamenáme si naměřenou hodnotu.
      \item
        Celí proces zopakujeme i pro rezistor R$_2$.
    \end{itemize}
   
  \subsection{Měření kapacity kondenzátoru C$_1$}
    \begin{itemize}
      \item
        Zapneme meřící prístroj RLCG BM 959 a nastavíme měřič na měření kapacity.
      \item
        Pomocí krokodýlků (chcete-li krokosvorek) propojíme kondenzátor s meřícím přístrojem.
      \item
        Chvíli počkáme než tento krásný muzejní kousek zobrazí korektní výsledek.
       \item
        Poznamenáme si naměřenou hodnotu.
       \item
        Vypneme měřící přístroj RLCG BM 959.
    \end{itemize}
  
  \subsection{Měření periody}
    \begin{itemize}
      \item
        Zapneme zdroj Z$_1$ a pomocí voltmetru V$_1$ nastavíme zadané napětí.
      \item
        Zapneme osciloskop a vhodně jej nastavíme.
      \item
        Připojíme osciloskop k pinu 3 a 1 IO555.
      \item
        Doladíme nastavení osciloskopu tak, aby perioda zabrala celou obrazovku a aby její krajní hodnoty byly v mřížce.
        \begin{itemize}
          \item
            Spočítáme si všechny čtverečky v ose $x$ které zabírá perioda a vynásobíme je časem který máme nastavený na jeden dílek.
          \item
            Spočítáme si všechny čtverečky v ose $x$ které zabírá část periody v úrovni H a vynásobíme je časem který máme nastavený na jeden dílek.
          \item
            Spočítáme si všechny čtverečky v ose $x$ které zabírá část periody v úrovni L a vynásobíme je časem který máme nastavený na jeden dílek.
        \end{itemize}
    \end{itemize}
  
  \subsection{Měření dob hran}
    \begin{itemize}
      \item
        Nastavíme si osu $x$ (časovou osu) tak, abychom vyděli přechod z úrovně L na úrověň H.
      \item
        Posuneme si průběh tak aby byl zarovnaný s čtverečkovým rastrem.
      \item
        Spočítáme si čtverečky osi $x$, které jsou v prostoru nástupné hrany a vynásobýme je časem, který máme nastavený ne jeden dílek.
      \item
        Vypočítanou hodnotu si poznačíme.
      \item
        Celí proces zopakujeme i pro sestuponu hranu. 
    \end{itemize}
  
  \subsection{Měření celkového rozkmitu (napětí špička špička)}
    \begin{itemize}
      \item
        Na osciloskopu nastavíme mód auto a poté nastavíme měření $U_{PP}$ (Peak to Peak - špička špička).
      \item
        Počkáme dokud osciloskop nezabrazí naměřenou hodnotu.
      \item
        Naměřenou hodnotu si zaznamenáme.
    \end{itemize}
    
  \subsection{Měření efektivní hodnoty}
    \begin{itemize}
      \item
        Na osciloskopu nastavíme mód auto a poté nastavíme měření $U_{AVG}$ (Average - průmer).
      \item
        Počkáme dokud osciloskop nezabrazí naměřenou hodnotu.
      \item
        Naměřenou hodnotu si zaznamenáme.
    \end{itemize}
    
  \subsection{Měření střední hodnoty}
    \begin{itemize}
      \item
        Na osciloskopu nastavíme mód auto a poté nastavíme měření $U_{RMS}$ (Root Mean Square).
      \item
        Počkáme dokud osciloskop nezabrazí naměřenou hodnotu.
      \item
        Naměřenou hodnotu si zaznamenáme.
    \end{itemize}
    
  \subsection{Měření maximální a minimální ustálené hodnoty}
    \begin{itemize}
      \item
        Na osciloskopu nastavíme mód auto.
      \item
        Počkáme dokud osciloskop nezabrazí průběh napětí.
      \item
        Z vykresleného průběhu zjistíme minumální a maximální ustálené napětí ($U_{MIN_{ustalena}}$, $U_{MAX_{ustalena}}$)
      \item
        Zaznamenáme si naměřené hodnoty.
    \end{itemize}
  
  \subsection{Měření časovacího kondenzátoru}
    \begin{itemize}
      \item
        Přepneme osciloskop nakanál B, který je připojen na pin 2 IO555.
      \item
        Nastavíme mód auto.
      \item
        Počkáme dokud osciloskop nevykreslí charakteristiku.
      \item
        Z vykreslené charakteristiky zjistíme niminální a maximální hodnotu napětí, a čas nabíjení a vybíjení kondenzátoru, přičemž postupujeme obdobně jako u měření obdélníhového výstupu na pinu 3.
      \item
        Naměření hodnoty si zaznamenáme.
      \item
        Vypneme osciloskop. 
    \end{itemize}
        
  \subsection{Měření frekvence}
    \begin{itemize}
      \item
        Zapneme čítač a nastavíme si vhodný rozsah.
      \item
        Připojíme čítač k pinu 3 a 1 IO555.
      \item
        Chvíli počkáme a zaznačíme si nameřenou hodnotu.
      \item
        Vypneme čítač.
      \item
        Vypneme napájecí zdroj Z$_1$.
      \item
        Ukončíme měření. 
    \end{itemize}
