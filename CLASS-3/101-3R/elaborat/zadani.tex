\begin{minipage}[H][9cm][c]{0.8\textwidth}
  \begin{enumerate}
    \item
      Pomocí IO555 navrhněte generátor obdélníkového napětí s těmito parametry:
      \newline
      $f = 5~kHz$, DCL (střída) $= 60~\%$, $U_{MIN_{ustalena}} = 0~V$, $U_{MAX_{ustalena}} = 6~V$
    \item
      Jmenovité hodnoty použitých součástek změřte a vypočítejte absolutní a procentní chyby měření.
    \item
      Generátor sestavte pomocí přípravku a změřte tyto parametry výstupního napětí:
      \newline
      periodu, frekvenci (pomocí $4\frac{1}{2} DMM$), dobu úrovně H a dobu úrovně L, DCL, střídu, dobu náběžné hrany, dobu sestupné hrany, maximální a minimální ustálenou hodnotu, celkový rozkmit, překmit úrovně H, střední hodnotu, efektivní hodnoru
    \item
      Změřte dobu nabíjení a vybíjení kondenzátoru, minimální a maximální hodnotu napětí na časovacím kondenzátoru.
    \item
      Na PC zakreslete časový průběh výstupního napětí generátoru, napětí na časovacím kondenzátoru.  
  \end{enumerate}

\end{minipage}


