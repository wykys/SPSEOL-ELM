\section{Závěr}
  \subsection{Chyby měřících přístrojů}
    \indent\indent
    Procentuální chyba milovoltmetru se pobybovala v intervalu $<\pm 3~\%;~\pm 5~\%>$. Tento měřící přístoj tety nespadá do kategorie těch nejpěsnější, nicměně lepší přístoj pro malá napětí  o vysokých frekvencích jsme neměli k dospozici.
  
  \subsection{Zhodnocení}
    \begin{enumerate}
      \item
        V úvodu jsem shrnul základní poznatky o paralerním a sériovém rezonačním obvodu, takže bych měl mít bod 1 splněný.
      \item
        Změřil jsem frekvenční charakteristiku sériového rezonančního obvodu v rozsahu $\pm 50~kHz$ kolem rezonanční frekvence $f_0$. Naměřené charakteristika není plně kompatibilní s teoretickým modelem tohoto zapojení. To může být způsobeno chybou měření nebo porazitními vlastnostmi použitého přípravku.
      \item
        Odpor $R_L$ byl spočítán s využitím vztahu (3). Jeho hodnota byla výpočtem určena na $196,9~\Omega$. Tata hodnota všek nemůže být povačována za správnou, protože jsem musel odvodit vztah ve kterém se nebude vyskytovat $C$. To bylo prakticky i teoreticky nemožné, tak sem jeho hodnotu $X_C$ zanedbal v doufání, že bude mít kapacitu větčí než $1~\mu F$. Z této vypočítané hodnoty jsem pak s využitém odpozeného vztahu (4) určil teoretickou inukčnost cívky $L$. Tu jsem stanovil na $196,6~\mu H$.
      \item
        Tento bod nebyl realizovatelný, protože nám nebyla sdělen jmenovité hodnota kondenzátoru~$C$. 
      \item
        Změřil jsem frekvenční charakteristiku paralerního rezonančního obvodu v rozsahu $\pm 50~kHz$ kolem rezonanční frekvence $f_0$. Naměřená charakteristika také není plně kompatibilní s teoretickým modelem tohoto zapojení.
      \item
        Změřil jsem frekvenční charakteristiku paralerního rezonančního obvodu se sníženým činitelem jakosti v rozsahu $\pm 50~kHz$ kolem rezonanční frekvence $f_0$. Naměřené charakteristika opět není plně kompatibilní s teoretickým modelem tohoto zapojení.
      \item
        Z naměřených hodnot jsem vytvořil grafy, ve vektorovém formátu *.eps, což se dá ocenit zejména při elektronickém prohlížení dokumentu.
    \end{enumerate}
