\begin{minipage}[H][11.48cm][c]{0.8\textwidth}
  \begin{enumerate}
    \item
      Multimetrem zkontrolujte prahové napětí $U_{TO_{MP}}$, určete vývody měřené diody.
    \item
      Změřte a nakreslete voltampérové charakteristiky (VACH) dané diody v propustném směru až do proudu $20~mA$; z VACH stanovte prahové napětí $U_{TO_{ch}}$ při $I_F~=~20~mA$, které v charakteristice vyznačte. Z VACH odečtěte a graficky vyznačte katalogový údaj: napětí v propustném směru $U_F$.
    \item
Do VACH nakreslete pracovní zatěžovací přímku a určete pracovní bod diody pro $R_{SERIOVY}~=~270~\Omega$, $U_{ZDROJE}~=~5~V$, $I~=~16,16~mA$.
    \item
      Stanovte grafickopočetní metodou velikost diferenciálního odporu diody $r_d$ pro proudu $I_F~=~18~mA$. Odečet v grafu vyznačte.
    \item
      U všech měřených hodnot vypočítejte procentní chybu měření.
		\item
			Pomocí systému UNIMA změřte VACH danné diody v propustném směru.
	\end{enumerate}
\end{minipage}


