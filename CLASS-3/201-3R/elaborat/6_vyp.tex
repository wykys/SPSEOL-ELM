\section{Vzory výpočtů}
  
  Výpočet relativní procentuální chyby digitu:
  \begin{equation}
    \delta _{digit\%} = \dfrac{\pm digit}{MH} \cdot 100 = \dfrac{\pm 0,01}{0,5} \cdot 100 \doteq \underline{\underline{\pm 2~\%}}
    \nonumber
  \end{equation}
  
  Celková procentuální chyba:
  \begin{equation}
    \delta_{\%} = \pm\delta_{MH\%} \pm \delta_{digit\%} = \pm 0,8 \pm 2 \doteq \underline{\underline{\pm 2,8~\%}}
    \nonumber
  \end{equation} 
  
  
  Výpočet bodu zatěžovací přímky ležícího na ose x s využitím vztahu (3)
  \begin{eqnarray}
      P = [U; 0] = \underline{\underline{[5; 0]}}
      \nonumber
    \end{eqnarray}
    
  Výpočet bodu zatěžovací přímky ležícího na ose y s využitím vztahu (4)
  \begin{eqnarray}
      Q = \left[0; \frac{U}{R}\right] = \left[0; \frac{5}{270}\right] \doteq \underline{\underline{[0; 0,185]}}
      \nonumber
    \end{eqnarray}
    
    
  Výpočet diferenciálního odporu s využitím vztahu (5)  
  \begin{equation}
    r_d = \dfrac{\Delta U}{\Delta I} = \dfrac{0,2}{5\cdot 10^{-3}} = \underline{\underline{40~\Omega}}
    \nonumber
  \end{equation}
  
