\section{Postup měření}
  \subsection{Určený vývodů měřené diody}
    \begin{itemize}
      \item
        Zapojíme měřící obvod dle schématu č. 1.
      \item
      	Zjistíme proud procházející obvodem. Pokud obvodem proud neprochází, nebo je proud velmi malí, tak to znamená že dioda je buď přerušená a nebo je na kladný potenciál zdroje připojena katoda diody, v opačném případě je na kladný potenciál zdroje připojena anoda.
		\end{itemize}
		
	\subsection{Měření VACH}
    \begin{itemize}
      \item
        Zapojíme obvod dle schématu č. 2.
      \item
      	Na regulovatelném ss. zdroji nastavujeme napětí.
      \item
        Pomocí měřících přístrojů (voltmetru a ampérmetru), si zaznamenáváme naměřené hodnoty.  
      \item
        Naměřené hodnoty vyneseme ho souřadnicového systému a proložíme je křivkou.
		\end{itemize}
		
	\subsection{Měření VACH měřícím systémem UNIMA}
    \begin{itemize}
      \item
        Vzhledem ke komplexnosti a množství funkcí tohoto měřícího systému, postupuje dle dokumentace k tomuto přístroji.
      \item
      	Kvůli chybějící možnosti exportovat grafická výstup ze systému UMINA, po naměření VACH vytvoříme snímek obrazovky, který dále zpracujeme rastvovímy editory k tomuto účelu určenými.
		\end{itemize}
