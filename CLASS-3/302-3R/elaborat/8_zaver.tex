\section{Závěr}
  
%  \begin{tabular}[H]{rcrl}
%    $20~V$ & $\pm0,5\%$ z MH $\pm 1$ digit & $19.99~V$ & $0,01~V$ \\
%    $2~mA$ & $\pm0,8\%$ z MH $\pm 1$ digit & $1.999~V$ & $0,001~mA$ \\
%    $20~mA$ & $\pm0,8\%$ z MH $\pm 1$ digit & $19.99~V$ & $0,01~mA$ \\
%    $200~mA$ & $\pm1,5\%$ z MH $\pm 5$ digit & $199.9~V$ & $0,1~mA$
%  \end{tabular}
  
  \subsection{Chyby měřících přístrojů}
    \indent\indent
    Procentuální chyba použitých DMM MASTECH-MY64 se pobybovala v intervalu $\pm0,714~\%$ až $\pm 6,5~\%>$. Přičemž nejmenší procentuální chyby byly na rozsahu $20~V$ a největší chyby byli na rozsahu $200~mA$. Naměřeným hodnotám bych ale nedával moc velkou váhu, protože na všech DMM blikala signalizace vybité baterie.
  
  \subsection{Zhodnocení}
    \begin{enumerate}
      \item
        V uvodu jsem shrnul základní poznatky o Zenerovích a lavinových diodách. Přižemž jsem se snažil zaměřit na jejich rozdíly a principy funkce.
      \item
        Vytvořil jsem graf VACH s využitím vlastního scriptu v pythonu využívajícího knihovnu pylab. Do grafu jsem vyznačil hodnoty I$_{ZD_{MIN}}$, U$_{ZD_{MIN}}$ a U$_{ZD_{MAX}}$. I$_{ZD_{MAX}}$ nebylo vyznačeno, protože v zadání bylo jen několik bodů na VA charakteristice. I$_{ZD_{MAX}}$ se ale nachází až za hranicí kterou sme měli měřit.
      \item
        Nakreslil jsem schéma zapojení paralerního stabilizátoru se Zenerovou diodou a vyznačil hlavní veličiny.
      \item
        Navrhnul jsem velikost rezistoru R$_1$ a to jak výpočtem tak experimentálně pomocí snížování hodnoty reostaru. Druhou jmenovanou metodou jsem došel k hodnotě $84~\Omega$. Pomocí mého orientačního výpočtu jsem dočelk k hodnotě $85,7~\Omega$, tato hodnota se sice od naměřené hodnoty liší o $1,7~\Omega$, ale jako odhad je to velmi dobrý výsledek.
      \item
        Sestavil jsem stabilizátor dle zadání a změril jsem:
        \begin{enumerate}
          \item
            V zadaném režimu obvod stabilizuje výstupní napětí.    
          \item            
            Změnou výstupního proudu $\Delta$I$_2 = 10~mA$ se výstupní napětí změní o $\Delta$U$_{ZD}~=~0,36~V$.
          \item
            Změnou vstupního napětí $\Delta$U$_1 = 2~V$ se změní výstupní napětí o $\Delta$U$_2~=~0,08~V$
        \end{enumerate}
      \item
        Obvod je odolný vůči odpojení zátěže, protože proud po odpojení zatěžovazího rezistoru R$_Z$ nepřesáhne maximální nedestruktvní hodnotu danou výrobcě, proud ani nepřesáhne $70~mA$ na které byla dioda testována pri měření VACH. Při odpojené zátěži a napájecím napětí U$_1 = 10~V$ obvodem protéká proud I$_1 = 64~mA$ a na Zenerově diodě vzniká úbitek napětí $4,17~V$.
      
    \end{enumerate}
