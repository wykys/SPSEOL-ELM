\section{Postup měření}
  \subsection{Měření VACH Zenerovy diody}
  \begin{itemize}
    \item
      Zapojíme úlohu podle schématu č. 1. Ochranný rezistor R$_1$ dáme na maximum.
    \item
      Postupně zvyšujeme napětí zdroje Z$_1$. Pokud se nám ani při maximálním napětí zdroje nepodaří dusáhnou proudu I$_{MAX}$, tak snížíme hodnotu ochranného rezistoru R$_1$.
    \item
      Po nalezení vhodné hodnoty rezistoru R$_1$ stáhneme napájecí napětí U$_0$ na minimum  a můžeme začít meřit.
    \item
      Budeme postupně zvyšovat napětí droje Z$_1$, když proud I dosáhne hodnoty: (0, 1, 2, 3, 5, 7, 10, 20, 30, 50, 70)~$mA$ tak si zaznamenáme hodnotu napětí U$_{ZD}$.
   \end{itemize}
   
   \subsection{Měření mezních hodnot stabilizátoru}
   \begin{itemize}
    \item
      Zapojíme úlohu podle schématu č. 2.
    \item
      Nastavíme zdroj Z$_1$ napětí U$_1$ na $9~V$ a z ampérmetru M$_4$ zjistíme proud která nám protéká zátěží.
    \item
      Nastavíme zdroj Z$_1$ napětí U$_1$ na $11~V$ a z ampérmetru M$_4$ zjistíme proud která nám protéká zátěží.
     \item
      Nastavíme zdroj Z$_1$ napětí U$_1$ na takovou hodnotu, aby proud protékající zátěží I$_2$ byl roven $55~mA$, poté pomocí voltmetru M$_3$ zjistíme úbytek napětí na Zenerově diodě.
    \item
      Nastavíme zdroj Z$_1$ napětí U$_1$ na takovou hodnotu, aby proud protékající zátěží I$_2$ byl roven $45~mA$, poté pomocí voltmetru M$_3$ zjistíme úbytek napětí na Zenerově diodě.
    \item
      Z VACH zjistíme, jestli Zenerova dioda zvládne takový proud, který byl měřen ampérmětrem M$_1$. Pkud ano, odpojíme zátěž a změříme proud procházející Zenerovou diodou pomocí ampérmetru M$_1$ a napětí na Zenerově diodě pomocí voltmetru M$_3$.
  \end{itemize}
