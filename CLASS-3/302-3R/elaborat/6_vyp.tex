\section{Vzory výpočtů}
  
  Výpočet relativní procentuální chyby digitu:
  \begin{equation}
    \delta _{digit\%} = \dfrac{\pm digit}{MH} \cdot 100 = \dfrac{\pm 5 \cdot 0,1}{20} \cdot 100 = \underline{\underline{\pm 2,5~\%}}
    \nonumber
  \end{equation}
  
  Celková procentuální chyba:
  \begin{equation}
    \delta_{\%} = \pm\delta_{MH\%} \pm \delta_{digit\%} = \pm 1,5 \pm 2,5 = \underline{\underline{\pm 4~\%}}
    \nonumber
  \end{equation}
  
  Celková absolutní chyba:
  \begin{equation}
    \Delta I = \dfrac{\delta_\%}{100} \cdot MH = \dfrac{4}{100} \cdot 20 = \underline{\underline{\pm 0,8~mA}}
    \nonumber
  \end{equation}
  
  Ztrátový výkon na Zenerově diodě:
  \begin{equation}
    P_{ZD} = U_0I =  4,33 \cdot 20 = \underline{\underline{86,6~mW}}
    \nonumber
  \end{equation}
  
  Předřadný odpor R$_1$ vypočítáme z využitím vztahu (2), tento výpočet je jen orientační, protože předem neznáme jaký roud poteče Zenerovou diodou:
  \begin{equation}
    R_1 = \dfrac{U_1-U_2}{I_1} =  \dfrac{10-4}{70} \doteq \underline{\underline{85,7~\Omega}}
    \nonumber
  \end{equation}
