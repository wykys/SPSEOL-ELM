\begin{minipage}[H][11.48cm][c]{0.8\textwidth}
  \begin{enumerate}
    \item
      Zopakujte si základní veličiny magnetického pole. Uveďte v teoretickém úvodu metody měření magnetické indukce, magnetické intenzity a magnetického toku.
    \item
      Vypočítejte velikost intenzity magnetického pole uvnitř solenoidu:\newline
      $N = 315$, $l=0,149~m$, $I=1~A$.
    \item
			Stanovte výpočtem funkci $B=f(I)$ uvnitř solenoidu a proveďte cejchování Hallovy sondy $B=f(U_H)$.
		\item
			Vytvořte graf cejchovní křivky Hallovy sondy $B=f(U_H)$. Stanovte koeficienty $K_1$ a $K_2$ aproximační rovnice $B = U_H \cdot K_1 + K_2$.
    \item
      Změřte velikost magnetické indukce pomocí ocejchované Hallovi sondy na povrchu magnetického obvodu dynamického reproduktoru. Stanovte chybu měření.
    \item
      Vypočítejte s ověřte měřením magnetickou intenzitu a magnetický tok uvnitř cívky:\newline
      $N = 600$, $I = 1~mA$.
	\end{enumerate}
\end{minipage}


