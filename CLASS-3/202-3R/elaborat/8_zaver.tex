\section{Závěr}
  
%  \begin{tabular}[H]{rcrl}
%    $20~V$ & $\pm0,5\%$ z MH $\pm 1$ digit & $19.99~V$ & $0,01~V$ \\
%    $2~mA$ & $\pm0,8\%$ z MH $\pm 1$ digit & $1.999~V$ & $0,001~mA$ \\
%    $20~mA$ & $\pm0,8\%$ z MH $\pm 1$ digit & $19.99~V$ & $0,01~mA$ \\
%    $200~mA$ & $\pm1,5\%$ z MH $\pm 5$ digit & $199.9~V$ & $0,1~mA$
%  \end{tabular}
  
  \subsection{Chyby měřících přístrojů}
    \indent\indent
    Procentuální chyba použitých měřícího přístroje $M_1$ nepřekročila $1,9~\%$. Procentuální chyby měřícího přístroje $M_3$ dozáhla $12~\%$. Tato chyby už je poměrně značná a tudíž výsledné měření není nejpřesnějcí, vzhledem k vybyvení které jsem měl k dispozici ale hodnoty přesnější být nemohli. Měření nejvíce ovlivnil velká proud který obvodem procházel.
  
  \subsection{Zhodnocení}
    \begin{enumerate}
      \item
        V úvodu byli popsány veličiny magnetického pole, které byli v tomto měření ať už přímou či nepřímou metodou měřeny.
      \item
        Výpočtem jsem stanovil intenzitu magnetického pole uvnitř solenoidu, velikost magnetické intenzity $H \doteq 2114,094~Am^{-1}$.
      \item
      	Hallová sonda byla ocejchovaná.
      \item
        Z údajů získaných při cejchování Hallovy sondy byla vytvořen graf cejchovací křivky. Dále byli určeny koeficienty $K_1 \doteq 0,3935$ a $K_2 \doteq -2,9906$.
      \item
        Byla změřena magnetická indukce na povrchu permanentních magnetů z floppy mechaniky a z reproduktoru. a spočítány chyby měření. Výsledky jsou shrnuty v tabulce č. 3.
      \item
      	Byla změřena magnetické indukce cívky s železným jádrem. $B \doteq 94,95~mT$
    \end{enumerate}
