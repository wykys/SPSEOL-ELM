\section{Závěr}
  \subsection{Chyby měřících přístrojů}
    Relativní procentní chyby zvoleného veltmetru $V_1$ npřesáhla 1 \%, tudíž by se dal považovat na vyhovující danné úloze.
  
    Relativní procentní chyby voltmetru $V_2$ na použitých rezsazích nepřekročila 2 \%. Maximální relativní chyba 1,824 \% už je poměrně dost, ale pro orientační měření tato chyby postačí.
  
  \subsection{Zhodnocení}
    Napětí $U_1$ bylo měřeno voltmetrem $V_1$, hodnoty dokarují že použitý zdroj BD-125 dodával konstantní napětí.

    Napětí $U_2$ bylo měřeno voltmetrem $V_2$. Změřené hodnoty se mírně lišili od teoreticky spočítaných hodnot $U_{2V}$. To může způsobovat chyba použitého voltmetru.

    Rezistor $R_1$ a $R_2$ byl měřen DMM s funkcí $V_2$.

    $R_C$ (rezistory $R_1$ a $R_2$ zapojeny seriově) byl změřen stejně jako teoreticky spočítaný. Z toho by se dalo usuzovat, že meření bylo dostatečně přesné.

    Rezistor $R_3$ byl měřen DMM s funkcí $V_2$. Měření prokázalo že použitou odporovou dekádou lze použít pro dannou úlohu, protože měřením se dali naměřit hodnoty záteže, které vyžadovalo zadání.
