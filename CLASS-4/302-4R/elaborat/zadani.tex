\begin{minipage}[H][11.48cm][c]{0.8\textwidth}
  \begin{enumerate}
    \item
      Pro zadanou hodnotu $I_C$ vypočítejte hodnotu $R_C$ a zvolte nejbližší hodnotu z řady.
    \item
      Nastavte pracovní bod tranzistoru (měřením $U_{RC}$) a ve výstupním obvodu tranzistorového zesilovače ($U_{RC}$, $U_{CE}$, $U_{RE}$) ověřte platnost II. KZ.
    \item
			Pro $f = 1~kHz$, $R_E = 10~\Omega$ změřte $A_u$ zesilovače pomocí:
			\begin{enumerate}
				\item
					dvou milivoltmetrů
				\item
					dvoukanálového osciloskopu
			\end{enumerate}
			Změřené hodnoty porovnejte a vyberte měřící metodu s menší chybou měření.
    \item
      Změřte mezní kmitočty zesilovače $f_d$ a $f_h$, na $mm$ papír nakreslete orientační tvar frekvenční přenosové charakteristiky zesilovače.
    \item
      Pro $f = 1~kHz$ změřte vstupní odpor zesilovače.
		\item
			Vhodným způsobem změňte úroveň zpětné vazby a opakujte měření parametrů zesilovače viz. body zadání 3-4-5.
		\item
			Změřte a zakreslete časové průběhy napětí na vstupu a výstupu zesilovače $f~=~1~kHz$.
	\end{enumerate}
\end{minipage}


