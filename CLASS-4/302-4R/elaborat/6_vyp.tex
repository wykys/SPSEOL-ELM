\section*{Vzory výpočtů}
  
  Výpočet procentuální chyby:
  \begin{equation}
    \delta _{\%U_2} = \dfrac{\pm TP \cdot MR}{MH} = \dfrac{\pm 1,5 \cdot 10}{4} = \underline{\underline{\pm 3,75~\%}}
    \nonumber
  \end{equation}
  
  \hspace*{2cm}kde:\newline    
  	\hspace*{4cm}$\delta _{\%U_2}$ \dotfill procentuální chyba\hspace*{4cm}\newline
  	\hspace*{4cm}$TP$ \dotfill třída přesnosti\hspace*{4cm}\newline
  	\hspace*{4cm}$MR$ \dotfill měřící rozsah\hspace*{4cm}\newline
  	\hspace*{4cm}$MH$ \dotfill měřená hodnota\hspace*{4cm}\newline
  
  Proud bází:
  \begin{equation}
    \beta = \dfrac{I_C}{I_B} \Rightarrow I_B = \dfrac{I_C}{\beta} = \dfrac{10 \cdot 10^{-3}}{68} \doteq \underline{\underline{147,058~\mu A}}
    \nonumber
  \end{equation} 
  
  
  Proud stabilizačním děličem:
  \begin{equation}
    I_D = 10I_C = 10 \cdot 147,058 = \underline{\underline{1,47~mA}}
    \nonumber
  \end{equation}
  
  Hodnota rezistoru $R_1$:
  \begin{equation}
    R_1 = \dfrac{U_{R_1}}{I_D} = \dfrac{9,2}{1,47} \doteq \underline{\underline{6,258~k\Omega}}
    \nonumber
  \end{equation} 
  
  Hodnota rezistoru $R_2$:
  \begin{equation}
    R_2 = \dfrac{U_{R_2}}{I_D} = \dfrac{0,8}{1,47} \doteq \underline{\underline{544,217~\Omega}}
    \nonumber
  \end{equation} 
  
  Hodnota rezistoru $R_3$:
  \begin{equation}
    R_3 = \dfrac{U_{R_3}}{I_C} = \dfrac{5}{10} = \underline{\underline{500~\Omega}}
    \nonumber
  \end{equation} 
  
  Hodnota rezistoru $R_4$:
  \begin{equation}
    R_4 = \dfrac{U_{R_4}}{I_C} = \dfrac{0,1}{10} = \underline{\underline{10~\Omega}}
    \nonumber
  \end{equation} 
   
  Ověření platnosti II. KZ. pro výstupní obvod: 
  \begin{equation}
    U_{CC} = U_{R_3} + U_{R_4} + U_{CE} = 4,51 + 0,086 + 5,26 = \underline{\underline{9,859~V}}
  	\nonumber
  \end{equation} 
  
  Napěťový přenos:
  \begin{equation}
  		A_u = \dfrac{U_{OUT}}{U_{IN}} = \dfrac{140}{4} = \underline{\underline{35}}
  		\nonumber
  \end{equation}
