\section*{Závěr}
  
%  \begin{tabular}[H]{rcrl}
%    $20~V$ & $\pm0,5\%$ z MH $\pm 1$ digit & $19.99~V$ & $0,01~V$ \\
%    $2~mA$ & $\pm0,8\%$ z MH $\pm 1$ digit & $1.999~V$ & $0,001~mA$ \\
%    $20~mA$ & $\pm0,8\%$ z MH $\pm 1$ digit & $19.99~V$ & $0,01~mA$ \\
%    $200~mA$ & $\pm1,5\%$ z MH $\pm 5$ digit & $199.9~V$ & $0,1~mA$
%  \end{tabular}
  
  \subsection*{Chyby měřících přístrojů}
    \indent\indent
    Procentuální chyba použitých měřících přístrojů ($M_1$, $M_2$, $M_3$ a $M_4$) nepřekročila $\pm5~\%$. tudíž by jse měření dalo považivat za relativně přesné. Největší procentuální chyby jsem se dopustil pši měření maximální hodnoty maxímálního napětí osciloskopem ($M_4$), chyby byly vypočítána na $\pm4,5~\%$.
  
  \subsection*{Zhodnocení}
    \begin{enumerate}
      \item
        Pro zadanou hodnotu $I_C = 10~mA$ jsem spočítal hodnotu rezistoru $R_3 = 500~\Omega$.
      \item
        Nyvrhoval jsem zesilovač pracující v třídě A. Pracovní bod tranzistoru jsem tedy nastavil téměř na polovinu $U_{CC}$, tedy $5~V$. Dosádnou přesné hodnoty poloviny $U_{CC}$ jse mi ale nepodařilo, jelikož mi v tom bránily tolerance součástek.
      \item
      	Změřil jsem hodnotu $A_u$, a to pomocí dvou nepřímích metod. Mření pomocí milivoltmetrů mělo menchí procentuální chybu (maximálně $\pm 3,75~\%$) než měření s osciloskopem. Naměřená hodnota pomocí milivoltmetru $A_u = 67,5$. Pomocí osciloskopu $A_u = 70$.
      \item
        Dle postupu uvedením v postupu jsem změřil horní a dolí mezní frekvenci zesilovače. $f_d = 241~Hz$ a $f_h = 589,712~kHz$. Na milimetrový papír jsem zakreslil orientační tvar přenosové charakteristiky.
      \item
        Vstupní odpor zesilovače byl stanoven na hodnotu $r_{VST} = 440~\Omega$. Docěla mě překvalila jeho moněrně nízká hodnota.
      \item
      	Změnil jsem hodnotu spětné vazby odpojením kondenzátoru $C_3$. Což mělo za následek rapidní sníření napěťového zenosu. Z hodnoty $A_u = 67,7$ jse hodnota změnila na $A_u = 35$. Dále jsem změřil horní a dolní mezní frekvenci a vstupní odpor, nyměření hodnoty jsou shrnuty v tabulce č. 5.
      \item
      	Na milimetrový papír jsem zakreslil časové půběhy vstupních a výstupních signálů a to jak s připojeným tak i odpojeným kondenzátorem $C_3$.
    \end{enumerate}
