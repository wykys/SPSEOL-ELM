\section*{Postup měření}
  \subsection*{Měření proudového zesilovacího činitele}
    \begin{itemize}
      \item
        DMM nastavíme na měření $\beta$.
      \item
      	K DMM vhodným způsobem připojíme zkoumaný tranzistor, viz schéma č. 2.
      \item
      	Z displaye DMM zjistíme naměřenou hodnotu.
		\end{itemize}
		
  \subsection*{Měření napěťového přenosu}
    \begin{itemize}
      \item
        Zapojíme obvod dle schématu č.3.
      \item
      	Na výstup generátoru nastavíme frekvenci $f = 1~kHz$.
      \item
      	Z milivoltmetrů a osciloskopu zjistíme měřené hodnoty.
      \item
      	Z naměřených hodnot dopočítáme $A_u$
		\end{itemize}
		
  \subsection*{Měření mezních kmitočtů zesilovače}
    \begin{itemize}
      \item
      	Zapojíme obvod dle schématu č.3.
      \item
      	Na výstup generátoru nastavíme frekvenci $f = 1~kHz$.
      \item
      	Změříme hodnoty vstupu a výstupu pomocí milivoltmetrů $M_1$ a $M_2$.
      \item
      	Ladíme frekvenci generátoru směrem nadoru, dokud úroveň na výstupu nepoklesne o tři decibely.
      \item
      	Zjistíme aktuální frekvenci pomocí čítače.
      \item
      	Opakujeme poslední dva body, ale frekvenci měníme směrem dolů.
		\end{itemize}
		
  \subsection*{Měření vstupního odporu}
    \begin{itemize}
      \item
        Zapojíme obvod dle schématu č. 4.
      \item
        Na generátoru nastavíme frekvenci $f = 1~kHz$.
      \item
      	Dekádu nastavýme na nejmenší možný odpor.
      \item
      	Změříme výstupní namětí, pomocí milivoltmetru $M_2$.
      \item
      	Zvyšujeme odpor dekády, dokud výstupní napětí neklesne na polovinu napětí naměřeného v minulém bodu.
      \item
      	Z dekády odečteme vstupní odpor.
      \item
      	Dekádu zapojíme dle schématu č. 5.
      \item
      	Měřením můžeme ověřit odpor dekády.
		\end{itemize}
