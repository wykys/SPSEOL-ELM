\section*{Vzory výpočtů}
  
  Výpočet odporu rezistoru $R_1$ provádím dosazením do vztahu (1):
  \begin{equation}
    R_1 = \dfrac{1}{2\pi f_M C_1} = \dfrac{1}{2\pi \cdot 1,5 \cdot 10^3 \cdot 150 \cdot 10^{-9}} \doteq \underline{\underline{707~\Omega}}
    \nonumber
  \end{equation}
  
  Výpočet odporu rezistoru $R_2$ provádíme dosazením do upraveného vztahu (2)před výpočtem je třeba určit odpor rezistoru $R_3=\underline{\underline{10~k\Omega}}$:
  \begin{equation}
    R_2 = (10^{\frac{a_U}{20}}-1)\cdot R_3 = (10^{\frac{10}{20}}-1)\cdot 10^4 \doteq \underline{\underline{22~k\Omega}}
    \nonumber
  \end{equation}   
  
  Výpočet procentuální chyba milivoltmetru:
  \begin{equation}
    \delta _{\%U_2} = \dfrac{\pm TP \cdot MR}{MH} = \dfrac{\pm 4 \cdot 1}{1} = \underline{\underline{\pm 1~\%}}
    \nonumber
  \end{equation} 
    
  Výpočet fázového posunu:
  \begin{equation}
    \varphi = \arcsin(\dfrac{1}{Y_1} \cdot Y_2) = \arcsin(\dfrac{1}{14} \cdot 10) \doteq \underline{\underline{45^\circ 35^\prime 5 ^{\prime \prime}}}
  	\nonumber
  \end{equation}
 