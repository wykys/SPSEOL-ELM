\section*{Závěr}
  
%  \begin{tabular}[H]{rcrl}
%    $20~V$ & $\pm0,5\%$ z MH $\pm 1$ digit & $19.99~V$ & $0,01~V$ \\
%    $2~mA$ & $\pm0,8\%$ z MH $\pm 1$ digit & $1.999~V$ & $0,001~mA$ \\
%    $20~mA$ & $\pm0,8\%$ z MH $\pm 1$ digit & $19.99~V$ & $0,01~mA$ \\
%    $200~mA$ & $\pm1,5\%$ z MH $\pm 5$ digit & $199.9~V$ & $0,1~mA$
%  \end{tabular}
  
  \subsection*{Chyby měřících přístrojů}
    \indent\indent
    Procentuální chyby použitých nf. milivoltmetrů byla maximálně $\pm10~\%$, což je už poměrně značná nepřesnost. Procentuální chyba použitého osciloskopu byla $\pm5~\%$.
  
  \subsection*{Zhodnocení}
    \begin{enumerate}
      \item
        Narvhnul sem dolnopropustný aktivní filtr s operačním zesilovačem a integračním článkem pro zadanou mezní frekvenci.
      \item
        Metodou bod po bodu byli změřeny frekvenční charakteristiky a naměřené hodnoty byli shrnuty v tabulce č. 2 a v grafech č. 1 a 2. Při měření bylo zjištěno že pokud k pasivnímu filtru připojíme OZ, tak se frekvenční charakteristika stává za mezní frekvencí více lineární. Dále bylo zjištěno, že filtr s OZ je účinnější, protože díky zesílení OZ je větší napěťový rozdíl mezi signály o frekvencí žádaných a o frekvencích nežádoucích
      \item
      	Změřená přenosová charakteristika byla vynesena go grafu vytvořeném v s pomocí knihovny pylab. Z důvodu větší přehlednosti jsem je ale rozdělil do dvou grafů.
      \item
        Z experimentálních důvodů jsem se rozhodl měřit frekvenční posun pomocí metody Lissajousových obrazců. Přestože u této metody vzniká poměrně veliká nepřesnost způsobená špatným vystředěním měřeného obrazce, tak se fázové posunu změřené pomocí této metody podobali hodnotám které jsem pro ověření změřil klasickou cestou. Fázové posuny jsou shrnuty v tabulce č. 3 a pomocí modelování signálů zobrazeny v grafu č. 3.
  \end{enumerate}
