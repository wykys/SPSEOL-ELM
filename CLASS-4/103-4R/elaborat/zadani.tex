\begin{minipage}[H][11.48cm][c]{0.8\textwidth}
  \begin{enumerate}
    \item
      Navrhněte aktivní doplnopropustný filtr 1. řádu s neinvertujícím zapojením OZ a RC členem, je-li požadována mezní frekvence $f_M=1,5~kHz$ a napěťový přenos OZ $a_{U_{oz}}=10~dB$. Kapacitu RC členu volte $C_1=150~nF$.
    \item
      Metodou bod po bodu změřte přenosovou frekvenční charakteristiku $a_U=f(f)$ v pásmu od $f_{MIN}=100~Hz$ do $f_{MAX}=100~kHz$.
      \begin{enumerate}
        \item
          RC členu
        \item
          RC členu a OZ
      \end{enumerate}
      Velikost vstupního napětí volte $1~V_{RMS}$.
    \item
			Změřené přenosové frekvenční charakteristiky $a_U=f(f)$ nakreslete na mm papír~/~na PC do jednoho grafu.   			
    \item
      Pro $f=2~kHz$ změřte fázový posun $\varphi$ mezi vstupním a výstupním napětím:
      \begin{enumerate}
        \item
          RC členu
        \item
          OZ
        \item
          celého filtru
      \end{enumerate}      
	\end{enumerate}
\end{minipage}


