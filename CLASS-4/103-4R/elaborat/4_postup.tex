\section*{Postup měření}
  \subsection*{Měření frekvenční přenosové charakteristiky metodou bod po bodu}
    \begin{itemize}
      \item
        Zapojíme obvod dle schématu č. 1.
      \item
      	Na výstup generátoru funkcí nastavíme kosinusové napětí $U_{RMS}=1~V$
      \item
      	Měníme frekvenci generovanou generátorem v rozsahu od $F_{MIN}$ do $f_{MAX}$, přičemž v okolí mezní frekvence měníme výstupní frekvenci po menších skocích než za nezní frekvencí nebo před ní.
      \item
      	Při každé změně frekvence si zapíšeme efektivní hodnotu napětí s měřidla $M_2$.
      \item
      	Tento postup opakujeme i pro schéma č. 2.
	\end{itemize}
		
  \subsection*{Měření fázového posunu metodou  Lissajousových obrazců}
    \begin{itemize}
      \item
        Zapojíme obvod dle schématu č.3.
	  \item
	  	Na generátoru nastavíme frekvenci $f=2~kHz$.
      \item
      	Nastavíme osciloskop do módu X-Y.
      \item
      	Posuneme zobrazený obrazec tak, aby byl bodově souměrný podle ocejchovaného středu obrazovky.
      \item
      	Spočítáme kolik dílků zabírá polovina výšky zobrazeného obrazce
      \item
      	Spočítáme kolik dílků od středu obrazec protíná Y-novou osu.
      \item
      	S naměřených hodnot spočítáme fázový posun.
      \item
      	Měření opakujeme i pro schéma č. 3, 4 a 5.
	\end{itemize}
		
 
