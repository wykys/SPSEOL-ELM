\section*{Vzory výpočtů}
  
  Výpočet geometrického průměru kondenzátoru $C$:
  \begin{equation}
    C = \sqrt{C_1 \cdot C_2} = \sqrt{34,16 \cdot 32,1} \doteq \underline{\underline{33,09~nF}}
    \nonumber
  \end{equation}
  
  Výpočet odporu rezistoru $R_1$ a $R_2$ provádíme dosazením do upraveného vztahu (1) pro dolní frekvenci 
  \begin{equation}
    R = \dfrac{1}{2\pi f_0 C} = \dfrac{1}{2\pi \cdot 300 \cdot 33,09 \cdot 10^{-9}} \doteq \underline{\underline{16~k\Omega}}
    \nonumber
  \end{equation}   
  
  Výpočet odporu rezistoru $R_1$ a $R_2$ provádíme dosazením do upraveného vztahu (1) pro horní frekvenci 
  \begin{equation}
    R = \dfrac{1}{2\pi f_0 C} = \dfrac{1}{2\pi \cdot 7000 \cdot 33,09 \cdot 10^{-9}} \doteq \underline{\underline{689~\Omega}}
    \nonumber
  \end{equation}     
  
  Střední hodnota výstupního napětí $U_{OUT_{AV}}$:
  \begin{equation}
    U_{OUT_{AV}} = \dfrac{U_{OUT_{MAX}} - U_{OUT_{MIN}}}{2} = \dfrac{7,4 - 7,3}{2}  \underline{\underline{7,35~V}}
    \nonumber
  \end{equation} 
    
  Výpočet procentní chyba výstupního napětí:
  \begin{equation}
    \delta_f = \dfrac{U_{OUT_{MAX}} - U_{OUT_MIN}}{U_{OUT_{AV}}} \cdot 100 = \dfrac{7,4 - 7,3}{7,35} \cdot 100 \doteq \underline{\underline{1,36~\%}}
  	\nonumber
  \end{equation}
 