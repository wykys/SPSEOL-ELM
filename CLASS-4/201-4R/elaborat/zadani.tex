\begin{minipage}[H][11.48cm][c]{0.8\textwidth}
	\begin{enumerate}
		\item
			Navrhněte oscilátor s Wienovým článkem RC, operačním zesilovačem MA 741CN a žárovkovou stabilizací napětí, požadován rozsah frekvencí $f_{min} = 300~Hz$ až $f_{max} = 7~kHz$. Napájecí napětí volte $\pm 15~V$, $C_1 = C_2 = 220~nF$.
		\item
			Sestavte navržený oscilátor, nastavte jeho optimální režim a změřte:
			\begin{enumerate}
				\item        
					skutečný frekvenční rozsah, změřený s vypočítanými a zapojenými součástkami
				\item
					závislost velikosti výstupního napětí na frekvenci, v rozsahu fmin až fmax a sestrojte graf
			\end{enumerate}                
		\item
			Vypočítejte procentní chybu změny výstupního napětí oscilátoru při přelaďování frekvence a sestavte graf na PC.
		\item
			Změřte maximální frekvenci, při které ještě nedochází ke znatelné změně velikosti nebo tvaru výstupního napětí.     
	\end{enumerate}
\end{minipage}


