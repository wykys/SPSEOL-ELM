\section*{Závěr}
  
%  \begin{tabular}[H]{rcrl}
%    $20~V$ & $\pm0,5\%$ z MH $\pm 1$ digit & $19.99~V$ & $0,01~V$ \\
%    $2~mA$ & $\pm0,8\%$ z MH $\pm 1$ digit & $1.999~V$ & $0,001~mA$ \\
%    $20~mA$ & $\pm0,8\%$ z MH $\pm 1$ digit & $19.99~V$ & $0,01~mA$ \\
%    $200~mA$ & $\pm1,5\%$ z MH $\pm 5$ digit & $199.9~V$ & $0,1~mA$
%  \end{tabular}
  
  \subsection*{Chyby měřících přístrojů}
    \indent\indent
    Procentuální chyby použitého osciloskopu je $\pm 3~\%$. Kondenzátory jsme měřili pomocí měřiče RLC, jehož chyby se pohybuje okolo $1~\%$. Největší chybu obvodu jsme způsobili měřením odporu žárovky pomocí DMM, tento problém je popsán v teoretickém úvodu.
  
  \subsection*{Zhodnocení}
    \begin{enumerate}
      \item
        Navrhl jsem a zrealizoval oscilátor s Wienovým článkem. Pomocí výše uvedených vztahů jsem dopočítal hodnoty součástek.
      \item
        Navržený oscilátor jsem zrealizoval na měřícím přípravku. Změřil sem skutečný frekvenční rozsah a závislost výstupního napětí na nastavené frekvenci. Tyto údaje jsou shrnuty v tabulkách číslo 4.
      \item
      	Vypočítal jsme procentuální chybu výstupního napětí $\delta_f = 1,36~\%$. A vytvořil jsem na PC graf závislosti výstupního napětí na nastavené rezonanční frekvenci. V grafu jde krásně vidět propouštěné pásmo a také i prudký pokles výstupního napětí.
      \item
        Díle byla změřena maximální frekvence, při které je zapojení schopno fungovat tak, že na výstupu je nezkreslený kosinový napěťový průběh. Tyto frekvence jsou shrnuty v tabulce číslo 3. Maximální frekvence dosáhla hodnoty $7,042~kHz$.
  \end{enumerate}
