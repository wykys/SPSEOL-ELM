\section*{Postup měření}
  \subsection*{Zapojení obvodu}
    \begin{itemize}
      \item
      	U použitých kondenzátorů si spočítáme geometrický průměr jejich skutečných hodnot. $C = \sqrt{C_1 \cdot C_2}$.
      \item
      	Dopočítáme si hodnoty rezistorů dle vztahu číslo (1).
      \item
        Změříme si odpor vlákna žárovky.      
      \item
      	Zapojíme obvod dle schématu č. 1.      	
	\end{itemize}
		
  \subsection*{Měření skutečného frekvenčního rozsahu}
    \begin{itemize}
      \item
        K obvodu připojíme na výstup milivoltmetr a najdeme mezní frekvenci měněním hodnoty spřažených potenciometrů $R_1$ a $R_2$.
	  \item
	  	Poté si spočítáme jakému napětí na výstupu odpovídá pokles o $3~dB$.
      \item
      	Pomocí regulování hodnoty potenciometrů $R_1$ a $R_2$ nastavíme na výstup napětí které jsme spočítali.
      \item
      	Z osciloskopu odečteme horní a dolní frekvenci.
	\end{itemize}

  \subsection*{závislost velikosti výstupního napětí na frekvenci}
    \begin{itemize}
      \item
        Postupně měníme hodnotu potenciometrů $R_1$ a $R_2$.
      \item
        Přitom odečítáme z osciloskopu výstupní napětí a výstupní frekvenci
      \item
        Získané údaje vyneseme do grafu.
    \end{itemize}
		
 
