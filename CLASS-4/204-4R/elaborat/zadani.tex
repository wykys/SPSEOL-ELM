\begin{minipage}[H][11.48cm][c]{0.8\textwidth}
	\begin{enumerate}
		\item
			Navrhněte a sestavte neinvertující nízkofrekvenční zesilovač s OZ: 741, je-li požadováno: napěťový přenos $a_U = 22~dB$ pro vstupní napětí $U_I = 0,6~V$ a frekvenci $f = 1~kHz$, napájecí napětí $\pm 15~V$, zatěžovací rezistor: $R_L = 10~k\Omega$, rezistor ve zpětné vazbě $R1 = 1~k\Omega$.
		\item
			Na sestaveném zesilovači změřte pomocí měřícího systému UNIMA:
			\begin{enumerate}
				\item        
					velikost výstupního harmonického napětí $U_0$ a z toho vypočítejte velikost skutečného napěťového přenosu $A_U$, $a_U$.
				\item
					maximální nezkreslený rozkmit výstupního napětí $U_{PP_{max}}$ při $f = 1~kHz$.
				\item
					znatelně omezený rozkmit výstupního napětí $U_{PP_{max}}$ při $f = 1~kHz$.
			\end{enumerate}                
		\item
			Navrhněte vazební kondenzátor CV tak, aby dolní mezní frekvence zesilovače byla $f_D = 150~Hz$ při $R = 10~k\Omega$. Přenosovou charakteristiku v rozsahu $10~Hz$ až $50~kHz$ zobrazte pomocí systému UNIMA.
		\item
			Na nepájivém poli sestavte dolní propust RC a napěťový sledovač s OZ. Obvod předřaďte kaskádově před nf zesilovač. $R = 227,4~k\Omega$, $C = 100~nF$, $f_H = 7~kHz$.  
		\item
			Pomocí UNIMY změřte přenosovou frekvenční charakteristiku obvodu z 4. bodu v rozsahu $f_{min} = 10~Hz$ až $f_{max} = 50~kHz$.  
	\end{enumerate}
\end{minipage}


