\section*{Závěr}
  
%  \begin{tabular}[H]{rcrl}
%    $20~V$ & $\pm0,5\%$ z MH $\pm 1$ digit & $19.99~V$ & $0,01~V$ \\
%    $2~mA$ & $\pm0,8\%$ z MH $\pm 1$ digit & $1.999~V$ & $0,001~mA$ \\
%    $20~mA$ & $\pm0,8\%$ z MH $\pm 1$ digit & $19.99~V$ & $0,01~mA$ \\
%    $200~mA$ & $\pm1,5\%$ z MH $\pm 5$ digit & $199.9~V$ & $0,1~mA$
%  \end{tabular}
  
  \subsection*{Chyby měřících přístrojů}
    \indent\indent
    Uvedená chyba měřícího přístroje UNIMA při měření pomocí módu digitální osciloskop by měla být menší než $1,5~\%$. Z důvodu použití kapacitních dekád, jsem byl limitován omezeným počtem jmenovitých hodnot, na které jsem musel zaokrouhlit hodnoty použitých kondenzátorů. Procentuální chyby osciloskopu GAS-620 je $\pm~3\%$. Tato chyby ovšem není konečná, jelikož je třeba uvažovat i chybu pozorovatele.
  
  \subsection*{Zhodnocení}
    \begin{enumerate}
      \item
        Navrhl jsem zesilovač, dle zadaných parametrů. Tento zesilovač byl realizován na měřícím přípravku, takže k němu stačilo připojit jen vhodné dekády.
      \item
        Změřil jsem výstupní napětí, které je zobrazeno v grafu číslo 1. a shrnuto v tabulce číslo 3. Dále jsem spočítal napěťový přenos, který vyšel $A_U = 6$. S tohoto napěťového přenosu byl ještě spočten napěťový zisk zesilovače $a_u = 15,59~dB$. Dále byl změřeno špičkové napětí, při kterém obvod začínal deformovat vstupní signál.
      \item
      	Dále jsem navrhl a na kontaktním poli zrealizoval napěťová sledovač, horní a dolní RC propust.
      \item
        Obvody s předchozího bodu byli připojeny k měřícímu přípravku, a následně byla změřena frekvenční přenosová charakteristika.
      \item
        V posledním bodu byla změřena frekvenční charakteristika celého obvodu. Tato charakteristika je v grafu číslo 3.
  \end{enumerate}
