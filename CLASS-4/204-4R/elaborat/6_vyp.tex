\section*{Vzory výpočtů}
  
  Výpočet zpětnovazebného rezistoru $R_2$ s použitím vztahu (2)
  \begin{equation}  	
    R_{2} = R_1 (10^{\frac{a_u}{20}} - 1) = 1000 (10^{\frac{22}{20}} - 1) = \underline{\underline{11,59~k\Omega}}
    \nonumber
  \end{equation}
  
  Výpočet kondenzátoru do dolní propusti s využitím vztahu (1)
  \begin{equation}
    C = \dfrac{1}{2\pi f_0 R} = \dfrac{1}{2\pi \cdot 150 \cdot 10^4} \doteq \underline{\underline{106,1~nF}}
    \nonumber
  \end{equation}   


  Výpočet napěťového zisku $A_U$
  \begin{equation}
    A_U = \dfrac{U_{O_{MAX}}}{{U_{I_{MAX}}}} = \dfrac{3}{0,5} = \underline{\underline{6}}
    \nonumber
  \end{equation} 

  
  Výpočet napěťového zisku $a_U$:
  \begin{equation}
    a_U = 20\log A_U = 20\log 6 \doteq \underline{\underline{15,56~dB}}
    \nonumber
  \end{equation}     
 