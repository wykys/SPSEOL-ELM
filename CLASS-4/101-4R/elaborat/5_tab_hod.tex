\section*{Tabulky naměřených a vypočítaných hodnot} 
  
 
  
    \begin{table}[H]
    \begin{center}
      \begin{tabular}[H]{!{\vrule width 1pt}c|c|c!{\vrule width 1pt}}
        \specialrule{1pt}{0pt}{0pt} 
        č.měření & $f~[kHz]$ & $U_O~[V]$	\\\specialrule{1pt}{0pt}{0pt}         
       	1	&	1,00	&	1,50\\\hline
        2	&	1,10	&	2,10\\\hline
        3	&	1,20	&	2,80\\\hline
        4	&	1,30	&	3,20\\\hline
        5	&	1,40	&	3,40\\\hline
        6	&	1,50	&	3,10\\\hline
        7	&	1,60	&	3,00\\\hline
        8	&	1,70	&	2,80\\\hline
        9	&	1,80	&	2,70\\\hline
        10	&	1,90	&	2,40\\\hline
        11	&	2,60	&	1,00

		\\\specialrule{1pt}{0pt}{0pt} 
        
      \end{tabular}
      
      \caption{Měření závislosti $A_u = f(f)$. Měřeno při $U_I =1~V \Rightarrow A_u$ má stejnou hodnotu jako $U_O$, jen nemá jednotku, proto jej v tabulce neuvádím. Chyba pro měření milivoltmetrem je uvedena v dokumentaci, $\delta_{mv\%} = \pm6\%$.}
      \label{tab:s1}      
    \end{center}
  \end{table}
  
  
  \begin{table}[H]
    \begin{center}
      \begin{tabular}[H]{!{\vrule width 1pt}c|c|c!{\vrule width 1pt}}
        \specialrule{1pt}{0pt}{0pt} 
        název měřené veličiny & měřená veličina & naměřená hodnota	\\\specialrule{1pt}{0pt}{0pt}         
       	dolní frekvence zesilovače			&	$f_D$	&	$1,1478~kHz$		\\\hline
        rezonanční frekvence zesilovače		&	$f_o$	&	$1,47675~kHz$		\\\hline
        horní frekvence zesilovače			&	$f_H$	&	$1,9062~kHz$		\\\hline
        šířka pásma							&	$B$		&	$758,4~Hz$			

		\\\specialrule{1pt}{0pt}{0pt} 
        
      \end{tabular}
      
      \caption{Důležité frekvenční parametry zesilovače.}
      \label{tab:s1}      
    \end{center}
  \end{table}
  
  
  \begin{table}[H]
    \begin{center}
      \begin{tabular}[H]{!{\vrule width 1pt}c|c|c!{\vrule width 1pt}}
        \specialrule{1pt}{0pt}{0pt} 
        název měřené veličiny & měřená veličina & naměřená hodnota	\\\specialrule{1pt}{0pt}{0pt}         
       	perioda						&	$T$			&	$678~us$			\\\hline
        oscilační frekvence			&	$f_o$		&	$1,47675~kHz$		\\\hline
        maximální hodnota napětí	&	$U_{MAX}$	&	$4~V$				\\\hline
        rozkmit napětí				&	$U_{PP}$	&	$7,84~V$			\\\hline
        efektivní hodnota napětí	&	$U_{RMS}$	&	$2,77~V$			\\\hline
        střední hodnota napětí		&	$U_{AVG}$	&	$17,3~mV$

		\\\specialrule{1pt}{0pt}{0pt} 
        
      \end{tabular}
      
      \caption{Měření základních parametrů oscilátoru pomocí digitálního osciloskopu.}
      \label{tab:s1}      
    \end{center}
  \end{table}
  
  
  \begin{table}[H]
    \begin{center}
      \begin{tabular}[H]{!{\vrule width 1pt}c!{\vrule width 1pt}c|c|c|c|c|c|c|c|c!{\vrule width 1pt}}
        \specialrule{1pt}{0pt}{0pt}        
       	$t~[us]$		&	0		&	100		&	200		&	300		&400	&500	&600	&	700		&800 \\\hline
        $u_{osc}~[V]$	&	0		&	2		&	4		&	2		&-2		&-4		&-3		&	0		&3	\\\hline
        $u_{vyp}~[V]$	&	0,000	&	3,201	&	3,839	&	1,403	&-2,156	&-3,989	&-2,628	&	0,837	&3,632

		\\\specialrule{1pt}{0pt}{0pt} 
        
      \end{tabular}
      
      \caption{Měřené a teoretické časové průběhy napětí.}
      \label{tab:s1}      
    \end{center}
  \end{table}
  
  
  \begin{table}[H]
    \begin{center}
      \begin{tabular}[H]{!{\vrule width 1pt}c|c!{\vrule width 1pt}}
        \specialrule{1pt}{0pt}{0pt} 
        	název frekvence & $f~[kHz]$	\\\specialrule{1pt}{0pt}{0pt} 
        	$f_1$		&	1,4768	\\\hline
        	$f_2$		&	1,4805	\\\hline
        	$f_3$		&	1,4769	\\\hline
        	$f_4$		&	1,4768	\\\hline
        	$f_5$		&	1,4769	\\\hline
        	$f_6$		&	1,4768	\\\hline
        	$f_7$		&	1,4773	\\\hline
        	$f_8$		&	1,4768	\\\hline
        	$f_9$		&	1,4768	\\\hline
        	$f_{10}$	&	1,4769	\\\hline
        	$f_{prum}$	&	1,4773

		\\\specialrule{1pt}{0pt}{0pt} 
        
      \end{tabular}
      
      \caption{Měření časové stability frekvence.}
      \label{tab:s1}      
    \end{center}
  \end{table}
  
 