\section*{Postup měření}
  \subsection*{Analýza obvodu}
    \begin{itemize}
      \item
        Důkladně si prohlédneme měřící přípravek a identifikujeme jednotlivé součástky.
      \item
      	Nakreslíme si schéma zapojení. Pokud bude přípravek promiňte mi ten výraz ,,nasoplený'' jako většina přípravků, neobejdete se při analyzování obvodu bez DMM.
      \item
      	Když máme schéma nakreslené, tak ještě dle prvního budu zadání doplníme hodnoty součástek.
	\end{itemize}
		
  \subsection*{Měření frekvenční závislosti $A_u = f(f)$}
    \begin{itemize}
      \item
        Zapojíme obvod dle schématu č.1. a nastavíme $R_1$ a $R_4$ do poloh dle zadání.
	  \item
	  	Na generátoru nastavíme frekvenci frekvenci od které chceme začít obvod zkoumat.
      \item
      	Nastavíme si na generátoru požadovanou úroveň. V našem případě jsme zvolili jeden volt. Tuto úroveň budeme považovat za konstantní.
      \item
      	Zapojíme schéma dle schématu č.2.
      \item
      	Ladíme na generátoru frekvence a měříme pomocí milivoltmetru odezvu obvodu.
      \item
      	Když rafička dosáhne maxima, tak si zapíšeme mezní frekvenci a výstupní napětí.
      \item
      	Dle vztahu (3) si dopočítáme jaké napětí dosáhneme při poklesu o $3~dB$.
      \item
      	Ladíme generátor nad a pod mezní frekvenci až najdeme mezní frekvence. Poznáme je podle toho že milivoltmetr bude ukazovat námi spočítanou hodnotu napětí při poklesu o $3~dB$.
      \item
        Lazením generátoru si obvod proměříme v okolí propouštěného pásma.
	\end{itemize}
	
	\subsection*{Měření oscilátoru}
    \begin{itemize}
      \item
        Zapojíme obvod dle schématu č.3. a nastavíme $R_1$ a $R_4$ do poloh dle zadání.
	  \item
	  	Pomocí digitálního osciloskopu a čítače změříme požadované veličiny.     
	\end{itemize}
		
 
