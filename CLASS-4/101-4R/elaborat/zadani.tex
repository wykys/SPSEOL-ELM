\begin{minipage}[H][11.48cm][c]{0.8\textwidth}
  \begin{enumerate}
    \item
      Podle přípravku nakreslete schéma zapojení selektivního nf zesilovače, označte všechny součástky a jejich jmenovité hodnoty.
    \item
      Změřte a na $mm$ papír zakreslete závislost modulu přenosu na frekvenci $A_u = f(f)$ u selektivního nf zesilovače nastavte $P_1 = 7$, $P_2 = 7$. Určete $f_0$ a šířku pásma pro pokles zesílení o $3~dB$.      
    \item
			Realizujte oscilátor ze selektivního nf zesilovače vhodným propojením vstupu a výstupu. Pro $P_1 = 9$ a $P_2 = 7$ výstup na hranici limitace. Změřte tyto parametry výstupního napětí $u_o$ oscilátoru: periodu, frekvenci, rozkmit, efektivní a střední hodnotu.
    \item
      Nakreslete na $mm$ papír graf výstupního napětí $u_o = f(t)$ oscilátoru. Změřené hodnoty z osciloskopu ověřte výpočtem, krok času zvolte $\frac{T}{8}$.
    \item
      Změřte a vypočítejte krátkodobou stabilitu frekvence oscilátoru v časovém intervalu 10 minut.
   \end{enumerate}
\end{minipage}


